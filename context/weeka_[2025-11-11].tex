% !TeX root = ../main.tex

\section{Homework \#10 [2025-11-11]}

\begin{problem}
  Interpreting the Fermi liquid via a Schrieffer-Wolff transformation.
  We start from a translationally invariant, spin-1/2 fermion system
  \[
    \mathcal H = \mathcal H + \mathcal H_\text{int},
  \]
  where
  \[
    \mathcal H_0 = \sum_{k,\sigma} \epsilon_k c_{k\sigma}^\dagger c_{k\sigma},
    \quad
    \mathcal H_\text{int} = \frac1{2V} \sum_{k,k',q,\sigma,\sigma'} V_{q}
      c_{k+q,\sigma}^\dagger c_{k'-q,\sigma'}^\dagger
      c_{k'\sigma'} c_{k\sigma}.
  \]
  The Fermi sea $\ket|\textsf{FS}>$ is the ground state of $\mathcal H_0$,
  with all states $|k| < k_F$ filled. We assume $V_q$ is weak and smooth
  near $q = 0$.

  To do a canonical transformation for a Fermi liquid state, we want to
  integrate out (cancel out) \textbf{off-shell} excitations. This is different
  from previous cases, where the original interacting terms are canceled out
  entirely. Instead, for a Fermi liquid state, we write $H_\text{int}$
  schematically as
  \[
    \mathcal H_\text{int} = \mathcal H_\text{diag} + H_\text{off},
  \]
  where
  \begin{itemize}
    \item $\mathcal H_\text{diag}$ conserves the number of quasiparticles near
    the Fermi surface (i.e., forward scattering, density-density type).
    \item $\mathcal H_\text{off}$ mixes sectors with different numbers of
    particle-hole pairs (e.g., it creates or destroys particle-hole pairs
    relative to the Fermi sea).
  \end{itemize}
  Formally,
  \[
    \mathcal H_\text{diag} = \frac1{2V}
      \sum_{\substack{k,k',q\\\text{both $k$, $k^\dagger q$ near $k_F$}}}
      V_q c_{k+q}^\dagger c_{k'-q}^\dagger c_{k'} c_k,
  \]
  and $\mathcal H_\text{off}$ is the rest.
  \paragraph{Such separation can be formally done via the semi-classical
    variational method}
  A simplified version is written
  \[
    \mathcal H_\text{diag} = \frac1{2V}
      \sum_{\substack{k,k',q\\
        \text{both $k = |k+q| = |k'-q| = |k'| = k_F$, $q\to0$}}}
      V_q c_{k+q}^\dagger c_{k'-q}^\dagger c_{k'} c_k.
  \]
  So you can see $\mathcal H_\text{off}$ is just $\mathcal H_\text{int}$ with at
  least one momentum that is away from the Fermi surface. We can just keep it in
  the form of $\mathcal H_\text{int}$.

  We choose $\mathcal S$ such that
  \[
    \mathcal H_\text{off} + [\mathcal S, \mathcal H_0] = 0.
  \]
  This means $\mathcal S$ is chosen to cancel the leading-order off-diagonal
  part of $\mathcal H_\text{int}$ under the transformation.

  Then the transformed Hamiltonian becomes
  \[
    \mathcal H' = \mathcal H_0 + \mathcal H_\text{diag}
      + \frac12[\mathcal S, \mathcal H_\text{off}] + \mathcal O(V^3),
  \]
  where the new effective Hamiltonian is block-diagonal up to order $V^2$.

  Constructing $\mathcal S$ explicitly.

  The commutator $[\mathcal S, \mathcal H_0]$ acts as
  \[
    [\mathcal S, \mathcal H_0]
  = \sum_{\alpha\beta} S_{\alpha\beta} (\epsilon_\alpha - \epsilon_\beta)
    c_\alpha^\dagger c_\beta,
  \]
  so to satisfy $[\mathcal S, \mathcal H_0] = -\mathcal H_\text{off}$,
  we can write
  \[
    \mathcal S = \sum_{\alpha\beta}
    \frac{(\mathcal H_\text{off})_{\alpha\beta}}
      {\epsilon_\alpha - \epsilon_\beta}.
  \]
  This is the Schrieffer-Wolff generator, which mixes states differing in
  unperturbed energy. For the present problem, $\mathcal S$ connects a bare
  fermion state $c_{k\alpha}^\dagger \ket|\textsf{FS}>$ to configurations with
  one additional particle-hole pair.
  \paragraph{Now, answer the following questions.}
  \begin{enumext}
    \item Evaluate $[\mathcal S, c_{k\sigma}^\dagger]$;
    \item Transforming the creation operator - ``ressing'' the bare fermion.
    The transformed fermion creation operator is
    \[
      \tilde c_{k\sigma}^\dagger
    = \upe^{\mathcal S} c_{k\sigma}^\dagger \upe^{-\mathcal S}
    = c_{k\sigma}^\dagger + [\mathcal S, c_{k\sigma}^\dagger]
    + \frac12[\mathcal S, [\mathcal S, c_{k\sigma}^\dagger]] + \cdots
    \]
    to the first order in $V$
    \[
      \tilde c_{k\sigma}^\dagger \approx c_{k\sigma}^\dagger
    + [\mathcal S, c_{k\sigma}^\dagger].
    \]
    \item Compute $\tilde c_{k\sigma}^\dagger \ket|\textsf{FS}>$.
    \item Compute quasiparticle weight
    $Z_k \equiv
    |\braket<\textsf{FS}|\tilde c_k\tilde c_k^\dagger|\textsf{FS}>|^2$
    from the canonical transformation.
  \end{enumext}
\end{problem}
\begin{solution}\leavevmode
  \begin{enumext}
    \item Starting from the identity of $\mathcal S$,
    \[
      [\mathcal S, \mathcal H_0] = -\mathcal H_\text{off}.
    \]
    To get the matrix elements of $\mathcal S$, substitute the commutator above
    into the states $\bra<m|$ and $\ket|n>$ ($m \neq n$)
    \[
     -\braket<m|\mathcal H_\text{off}|n>
    = \braket<m|[\mathcal S, \mathcal H_0]|n>
    = \braket<m|\mathcal S|n> E_n - E_m \braket<m|\mathcal S|n>
    = (E_n - E_m) \braket<m|\mathcal S|n>,
    \]
    where $\bra<m|\mathcal H_0 = \bra<m| E_m$
    and $\mathcal H_0\ket|n> = E_n\ket|n>$. Then,
    \[
      \braket<m|\mathcal S|n>
    = \frac{\braket<m|\mathcal H_\text{off}|n>}{E_m - E_n}
    \qq{for} m \neq n,
    \]
    where we set the diagonal elements $\braket<n|\mathcal S|n> = 0$.
    By inserting $\ketbra|m><n|$ to form the identity $\identity$,
    the operator $\mathcal S$ in Dirac notation can be expressed as
    \[
      \mathcal S = \sum_m \sum_n \ket|m>
                   \frac{\braket<m|\mathcal H_\text{off}|n>}{E_m - E_n} \bra<n|
    \]
    So, we can generate $\mathcal S$ from the four operators in
    $\mathcal H_\text{off}$, which is equivalent to the four operators in
    $\mathcal H_\text{int}$ but without the diagonal elements, that is
    \[
      \mathcal S = \sum_{k, k', q, \sigma, \sigma'} S_{kk'q}^{\sigma\sigma'}
        c_{k+q, \sigma}^\dagger c_{k'-q, \sigma'}^\dagger
        c_{k'\sigma'} c_{k\sigma},
    \]
    Then, substitute it into the condition
    $[\mathcal S, \mathcal H_0] = -\mathcal H_\text{off}$ to
    determine $S_{kk'q}^{\sigma\sigma'}$.
    To distinguish the indices, we need to write $\mathcal H_0$ as
    \[
      \mathcal H_0 = \sum_{p, \tau} \epsilon_p c_{p\tau}^\dagger c_{p\tau}
    \]
    Then, the commutator is
    \[
      [\mathcal S, \mathcal H_0]
    = \sum_{k,k',q,\sigma,\sigma'} S_{kk'q}^{\sigma\sigma'}
      \sum_{p,\tau} \epsilon_p [c_{k+q,\sigma}^\dagger c_{k'-q,\sigma'}^\dagger
       c_{k'\sigma'} c_{k\sigma}, c_{p\tau}^\dagger c_{p\tau}].
    \]
    Compute the kernel first
    \begin{multline*}
      [c_{k+q,\sigma}^\dagger c_{k'-q,\sigma'}^\dagger
       c_{k'\sigma'} c_{k\sigma}, c_{p\tau}^\dagger c_{p\tau}]
    = [c_{k+q,\sigma}^\dagger, c_{p\tau}^\dagger c_{p\tau}]
      c_{k'-q,\sigma'}^\dagger c_{k'\sigma'} c_{k\sigma}
    + c_{k+q,\sigma}^\dagger
      [c_{k'-q,\sigma'}^\dagger, c_{p\tau}^\dagger c_{p\tau}]
      c_{k'\sigma'} c_{k\sigma}\\
    + c_{k+q,\sigma}^\dagger c_{k'-q,\sigma'}^\dagger
      [c_{k'\sigma'}, c_{p\tau}^\dagger c_{p\tau}] c_{k\sigma}
    + c_{k+q,\sigma}^\dagger c_{k'-q,\sigma'}^\dagger
      c_{k'\sigma'} [c_{k\sigma}, c_{p\tau}^\dagger c_{p\tau}].
    \end{multline*}
    The four terms
    \begin{enumext}[columns = 2]
      \item $[c_{k+q,\sigma}^\dagger, c_{p\tau}^\dagger c_{p\tau}]
    = -c_{p\tau}^\dagger \delta_{p,k+q} \delta_{\tau,\sigma}$
      \item $[c_{k'-q,\sigma'}^\dagger, c_{p\tau}^\dagger c_{p\tau}]
    = -c_{p\tau}^\dagger \delta_{p,k'-q} \delta_{\tau, \sigma'}$
      \item $[c_{k'\sigma'}, c_{p\tau}^\dagger c_{p\tau}]
    = c_{p\tau} \delta_{k'p} \delta_{\sigma'\tau}$
      \item $[c_{k\sigma}, c_{p\tau}^\dagger c_{p\tau}]
    = c_{p\tau} \delta_{kp} \delta_{\sigma\tau}$
    \end{enumext}
    where we used the anti-commutative properties of the Fermions
    \[
      \{c_{i,j}, c_{i',j'}^\dagger\} = \delta_{i,i'} \delta_{j,j'}, \quad
      \{c_{i,j}, c_{i',j'}\} = \{c_{i,j}, c_{i',j'}\} = 0.
    \]
    Substitute them into the second sum of the commutator
    \begin{align*}
      \sum_{p,\tau} \epsilon_p [c_{k+q,\sigma}^\dagger c_{k'-q,\sigma'}^\dagger
       c_{k'\sigma'} c_{k\sigma}, c_{p\tau}^\dagger c_{p\tau}]
= & - \mathemph{\sum_{p,\tau} \epsilon_p
      c_{p\tau}^\dagger \delta_{p,k+q} \delta_{\tau,\sigma}}
      c_{k'-q,\sigma'}^\dagger c_{k'\sigma'} c_{k\sigma}\\
  & - \mathemph{\sum_{p,\tau} \epsilon_p} c_{k+q,\sigma}^\dagger
      \mathemph[\sum_p]
        {c_{p\tau}^\dagger \delta_{p,k'-q} \delta_{\tau, \sigma'}}
      c_{k'\sigma'} c_{k\sigma}\\
  & + \mathemph{\sum_{p,\tau} \epsilon_p}
      c_{k+q,\sigma}^\dagger c_{k'-q,\sigma'}^\dagger
      \mathemph[\sum_p]
        {c_{p\tau} \delta_{k'p} \delta_{\sigma'\tau}} c_{k\sigma}\\
  & + \mathemph{\sum_{p,\tau} \epsilon_p}
      c_{k+q,\sigma}^\dagger c_{k'-q,\sigma'}^\dagger
      c_{k'\sigma'} \mathemph[\sum_p]{c_{p\tau}
        \delta_{kp} \delta_{\sigma\tau}}.
    \end{align*}
    Due to the sifting property of the $\delta$-function, the second sum of the
    commutator becomes (here take the minus sign out
    since $[\mathcal S, \mathcal H_0] = -\mathcal H_\text{off}$)
    \[
      \sum_{p,\tau} \epsilon_p [c_{k+q,\sigma}^\dagger c_{k'-q,\sigma'}^\dagger
       c_{k'\sigma'} c_{k\sigma}, c_{p\tau}^\dagger c_{p\tau}]
    = -(\epsilon_{k+q} + \epsilon_{k'-q} - \epsilon_{k'} - \epsilon_k)
      c_{k+q,\sigma}^\dagger c_{k'-q,\sigma'}^\dagger c_{k'\sigma'} c_{k\sigma}.
    \]
    Hence, the commutator $[\mathcal S, \mathcal H_0]$ becomes
    \[
      -\mathcal H_\text{off} = [\mathcal S, \mathcal H_0]
    = -\sum_{k,k',q,\sigma,\sigma'} S_{kk'q}^{\sigma\sigma'}
      (\epsilon_{k+q} + \epsilon_{k'-q} - \epsilon_{k'} - \epsilon_k)
      c_{k+q,\sigma}^\dagger c_{k'-q,\sigma'}^\dagger c_{k'\sigma'} c_{k\sigma}
    \]
    To compare with the expression of $\mathcal H_\text{off}$
    \[
      \mathcal H_\text{off} = \frac1{2V}
        \sum_{k,k',q,\sigma,\sigma'}
        V_{q}
        c_{k+q,\sigma}^\dagger c_{k'-q,\sigma'}^\dagger
        c_{k'\sigma'} c_{k\sigma} \qq{where} k\neq k',\ \sigma\neq\sigma'.
    \]
    Then, we arrive at
    \begin{multline*}
      \sum_{k,k',q,\sigma,\sigma'} S_{kk'q}^{\sigma\sigma'}
      c_{k+q,\sigma}^\dagger c_{k'-q,\sigma'}^\dagger c_{k'\sigma'} c_{k\sigma}
    = \mathcal S\\
    = \frac1{2V}
      \sum_{k,k',q,\sigma,\sigma'} \frac{V_{q}}
        {\epsilon_{k+q} + \epsilon_{k'-q} - \epsilon_k - \epsilon_{k'}}
      c_{k+q,\sigma}^\dagger c_{k'-q,\sigma'}^\dagger
      c_{k'\sigma'} c_{k\sigma}.
    \end{multline*}
    This is the second-quantized form of $\mathcal S$.
    Now, calculate the commutator $[\mathcal S, c_{k\sigma}^\dagger]$.
    Calculate the kernel
    \[
      [c_{k+q,\sigma}^\dagger c_{k'-q,\sigma'}^\dagger
       c_{k'\sigma'} c_{k\sigma}, c_{k\sigma}^\dagger]
    \]
    first. To distinguish the indices, we take
    $c_{k\sigma}^\dagger \to c_{p\tau}^\dagger$
    \begin{align*}
      [c_{k+q,\sigma}^\dagger c_{k'-q,\sigma'}^\dagger
       c_{k'\sigma'} c_{k\sigma}, c_{p\tau}^\dagger] &
    = c_{k+q,\sigma}^\dagger c_{k'-q,\sigma'}^\dagger(
      c_{k'\sigma'} \{c_{k\sigma}, c_{p\tau}^\dagger\}
    - \{c_{k'\sigma'}, c_{p\tau}^\dagger\}c_{k\sigma})\\
  & = c_{k+q,\sigma}^\dagger c_{k'-q,\sigma'}^\dagger
      c_{k'\sigma'} \delta_{kp} \delta_{\sigma\tau}
    - c_{k+q,\sigma}^\dagger c_{k'-q,\sigma'}^\dagger
      \delta_{k'p} \delta_{\sigma'\tau} c_{k\sigma},
    \end{align*}
    where $c_{p\tau}^\dagger$ only ``knock'' on $c_{k'\sigma'}$ and
    $c_{k\sigma}$ effectively due to the anti-commutative properties of
    the Fermions. We define the function
    \[
      \Phi(p_1,p_2,q) = \frac1
        {\epsilon_{p_1+q} + \epsilon_{p_2-q} - \epsilon_{p_1} - \epsilon_{p_2}}.
    \]
    Then, the commutator becomes
    \begin{align*}
      [\mathcal S, c_{p\tau}^\dagger] &
    = \sum_{k,k',q,\sigma,\sigma'} \frac{V_q\Phi(k,k',q)}{2V} (
      c_{k+q,\sigma}^\dagger c_{k'-q,\sigma'}^\dagger
      c_{k'\sigma'} \delta_{kp} \delta_{\sigma\tau}
    - c_{k+q,\sigma}^\dagger c_{k'-q,\sigma'}^\dagger
      \delta_{k'p} \delta_{\sigma'\tau} c_{k'\sigma'})\\
  & = \sum_{k',q,\sigma'} \frac{V_q\Phi(p,k',q)}{2V}
      c_{p+q,\tau}^\dagger c_{k'-q,\sigma'}^\dagger c_{k'\sigma'} -
      \sum_{k,q,\sigma} \frac{V_q\Phi(k,p,q)}{2V}
      c_{k+q,\sigma}^\dagger c_{p-q,\tau}^\dagger c_{k,\sigma}.
    \end{align*}
    Swap dummy variables in the second sum: $k \to k'$, $\sigma \to \sigma'$,
    $q \to -q$, and use
    $c_{p+q}^\dagger c_{k'-q}^\dagger = -c_{k'-q}^\dagger c_{p+q}^\dagger$
    \[
      [\mathcal S, c_{p\tau}^\dagger]
    = \sum_{k',q,\sigma'} \frac{V_q\Phi(p,k',q)}{2V}
      c_{p+q,\tau}^\dagger c_{k'-q,\sigma'}^\dagger c_{k'\sigma'} +
      \sum_{k,q,\sigma} \frac{V_{-q}\Phi(k',p,-q)}{2V}
      c_{p+q,\tau}^\dagger c_{k'-q,\sigma'}^\dagger c_{k'\sigma'},
    \]
    where $\Phi(p,k',q) = \Phi(k',p,-q)$. Hence,
    \[
      [\mathcal S, c_{p\tau}^\dagger]
    = \sum_{k',q,\sigma'} \frac{V_q + V_{-q}}{2V}
      \frac{c_{p+q,\tau}^\dagger c_{k'-q,\sigma'}^\dagger c_{k'\sigma'}}
        {\epsilon_{p+q} + \epsilon_{k'-q} - \epsilon_{p} - \epsilon_{k'}}, ~
      [\mathcal S, c_{k\sigma}^\dagger]
    = \sum_{k',q,\sigma'} \frac{V_q + V_{-q}}{2V}
      \frac{c_{k+q,\sigma}^\dagger c_{k'-q,\sigma'}^\dagger c_{k'\sigma'}}
        {\epsilon_{k+q} + \epsilon_{k'-q} - \epsilon_{k} - \epsilon_{k'}}.
    \]
    \item Substitute the result from (a) directly
    \[
      \tilde c_{k\sigma}^\dagger \approx c_{k\sigma}^\dagger
    + \sum_{k',q,\sigma'} \frac{V_q + V_{-q}}{2V}
      \frac{c_{k+q,\sigma}^\dagger c_{k'-q,\sigma'}^\dagger c_{k'\sigma'}}
        {\epsilon_{k+q} + \epsilon_{k'-q} - \epsilon_{k} - \epsilon_{k'}}.
    \]
    \item Substitute the result from (b) directly
    \[
      \tilde c_{k\sigma}^\dagger \ket|\textsf{FS}>
    = c_{k\sigma}^\dagger \ket|\textsf{FS}>
    + \sum_{k',q,\sigma'} \frac{V_q + V_{-q}}{2V}
      \frac{c_{k+q,\sigma}^\dagger c_{k'-q,\sigma'}^\dagger c_{k'\sigma'}}
        {\epsilon_{k+q} + \epsilon_{k'-q} - \epsilon_{k} - \epsilon_{k'}}.
      \ket|\textsf{FS}>
    \]
    \item Evaluate the overlap
    $\bra<\textsf{FS}|c_{k}\tilde c_{k}^\dagger\ket|\textsf{FS}>$ term by term
    \begin{enumext}
      \item The zeroth order $\mathcal O(V^0)$
      \[
        \braket<\textsf{FS}|c_kc_k^\dagger|\textsf{FS}> = 1.
      \]
      Here assume $|k| > k_F$, so $c_{k} c_{k}^\dagger = 1$ in the Fermi sea.
      \item Denote the operator
      \[
        \chi_k \equiv [\mathcal S, c_k^\dagger].
      \]
      The first order $\mathcal O(V^1)$
      \[
        \bra<\textsf{FS}|c_{k} \chi_k \ket|\textsf{FS}>
      = \braket<\textsf{FS}|c_{k} [\mathcal S, c_{k}^\dagger]|\textsf{FS}> = 0,
      \]
      where $\chi_k$ creates a \emph{two-particle, one-hole state} relative to
      the Fermi sea. Applying $c_k$ still leaves a net excitation, and the
      matrix element vanishes since the states are orthogonal
      ($\mathcal{S}$ is off-diagonal).
      \item The second order $\mathcal O(V^2)$
      \[
        \frac12 \braket<\textsf{FS}|c_{k} [\mathcal S, \chi_k]|\textsf{FS}>
      = \frac12 \braket<\textsf{FS}|
          c_{k} [\mathcal S, [\mathcal S, c_{k}^\dagger]]|\textsf{FS}>
      = -\frac12 \bra<\textsf{FS}| \chi_k^\dagger \chi_k \ket|\textsf{FS}>
      = -\frac12 |\chi_k \ket|\textsf{FS}>|^2.
      \]
      Since the generator $\mathcal{S}$ is anti-Hermitian
      ($\mathcal{S}^\dagger = -\mathcal{S}$), the second-order projection
      simplifies to $-\frac12 \braket<\chi_k^\dagger \chi_k>_{\text{FS}}$.
      Substituting the explicit form of $\chi_k$ into the weight, we obtain
      \[
        Z_k = 1 - \frac1{4V^2} \sum_{k',q,\sigma'}
          \frac{|V_q + V_{-q}|
          \theta(k_F - |k'|) \theta(|k + q| - k_F) \theta(|k' - q| - k_F)}
            {(\epsilon_{k+q} + \epsilon_{k'-q}
            - \epsilon_{k} - \epsilon_{k'})^2} < 1,
      \]
      which brings a quasiparticle weight less than $1$.
    \end{enumext}
  \end{enumext}
\end{solution}