% !TeX root = ../main.tex

\section{Homework \#9 [2025-11-04]}

\begin{problem}
  Calculate the Landau parameters to leading order in $\lambda_{1,2}$ for a
  Fermi liquid with the contact interactions
  \begin{enumext}
    \item $V(x - x') = \lambda_1\delta^{(3)}(x - x')$.
    \item $V(x - x') = -\lambda_2\nabla^2\delta^{(3)}(x - x')$
          (so that $V(q) = \lambda_1 q^2$ in Fourier space).
    \item Taking the results of (a) and (b) literally, sketch the regions of
          the $\lambda_1$, $\lambda_2$ phase diagram where the Fermi surface
          becomes unstable.
  \end{enumext}
\end{problem}
\begin{solution}\leavevmode
  \begin{enumext}
    \item 
  \end{enumext}
\end{solution}

\begin{problem}
  Test your understanding of Landau's mass renormalization formula by
  generalizing it to include the effect of a magnetization. Suppose we introduce
  a second vector potential into \eqref{6.86}
  \begin{equation}
    A(\theta) \underset{|\bm k|\to0}\sim \int_{k_F-k\cos\theta}^{k_F} \d q
    \frac{2\iu\pi a^2}{\omega - v_F(|\bm k + \bm q| - q)}
  = \frac{2\iu\pi a^2k\cos\theta}{\omega - v_Fk\cos\theta}.
    \tag{6.86} \label{6.86}
  \end{equation}
  that couples to the spin current, writing
  \[
    \mathcal H[\mathbf A_N, \mathbf W]
  = \sum_\sigma \int \d^3x \frac1{2m} \psi_\sigma^\dagger(x)
    [(-\iu\hbar\nabla - \mathbf A_N - \sigma\mathbf W)^2] \psi_\sigma(x)
  + \hat V
  \]
  Whereas $\mathbf A_N$ couples to the current of particles, $\mathbf W$
  couples to the ($z$ component of the) spin current.
  Assume that $V$ conserves spin current.
  \begin{enumext}
    \item By comparing the bare shift of the energies
    \[
      \delta\epsilon_{\mathbf p\sigma}^{(0)}
    = -\frac{\mathbf p}{m} \cdot (\mathbf A_N + \sigma \mathbf W)
    \]
    with the shift that result from interaction feedback,
    \[
      \delta\epsilon_{\mathbf p\sigma}
    = -\frac{\mathbf p}{m^*} \cdot \mathbf A_N
    - \sigma \frac{\mathbf p}{m_s^*} \cdot \mathbf W,
    \]
    show that there are two different mass renormalizations,
    \begin{align*}
      \frac{m}{m^*}   & = \frac1{1 + F_1^s},\\
      \frac{m}{m_s^*} & = \frac1{1 + F_1^a}.
    \end{align*}
    \item Show that, when the Fermi liquid is polarized, the masses of the
    ``up'' and ``down'' quasiparticles are now different, and given by
    \[
      \frac1{m_\sigma^*} = \frac1m\ab[\frac1{1 + F_1^s} + \frac M{1 + F_1^a}],
      \quad (\sigma = \uparrow, \downarrow)
    \]
    where the magnetization $M = n_\uparrow - n_\downarrow$ is the difference of
    ``up'' and ``down'' densities.
  \end{enumext}
\end{problem}
\begin{solution}\leavevmode
  \begin{enumext}
    \item 
  \end{enumext}
\end{solution}