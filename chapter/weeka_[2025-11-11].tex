% !TeX root = ../main.tex

\section{Homework \#10 [2025-11-11]}

\begin{problem}
  Interpreting Fermi liquid via a Schrieffer-Wolff transformation.
  We start from a translationally invariant, spin-1/2 fermion system
  \[
    \mathcal H = \mathcal H + \mathcal H_\text{int}
  \]
  where
  \[
    \mathcal H_0 = \sum_{k,\sigma} \epsilon_k c_{k\sigma}^\dagger c_{k\dagger},
    \quad
    \mathcal H_\text{int} = \frac1{2V} \sum_{k,k',q,\sigma,\sigma'} V_{q}
      c_{k+q,\sigma}^\dagger c_{k'-q,\sigma'}^\dagger
      c_{k'\sigma'} c_{k\sigma}.
  \]
  The Fermi sea $\ket|\text{FS}>$ is the ground state of $\mathcal H_0$,
  with all states $|k| < k_F$ filled. We assume $V_q$ is weak and smooth
  near $q = 0$.

  To do canonical transformation for a Fermi liquid state, we want to integrate
  out (cancel out) \textbf{off-shell} excitations. This is different from
  previous cases, where the original interacting terms are canceled out
  entirely. Instead, for a Fermi liquid state, We write $H_\text{int}$
  schematically as
  \[
    \mathcal H_\text{int} = \mathcal H_\text{diag} + H_\text{off}
  \]
  where
  \begin{itemize}
    \item $\mathcal H_\text{diag}$ conserves the number of quasiparticles near
    the Fermi surface (i.e., forward scattering, density-density type).
    \item $\mathcal H_\text{off}$ mixes sectors with different numbers of
    particle-hole pairs (e.g., it creates or destroys particle-hole pairs
    relative to the Fermi sea).
  \end{itemize}
  Formally,
  \[
    \mathcal H_\text{diag} = \frac1{2V}
      \sum_{\substack{k,k',q\\\text{both $k$, $k^\dagger q$ near $k_F$}}}
      V_q c_{k+q}^\dagger c_{k'-q}^\dagger c_{k'} c_k,
  \]
  and $\mathcal H_\text{off}$ is the rest.
  \paragraph{Such separation can be formally done via the semi-classical
    variational method}
  A simplified version is write
  \[
    \mathcal H_\text{diag} = \frac1{2V}
      \sum_{\substack{k,k',q\\
        \text{both $k = |k+q| = |k'-q| = |k'| = k_F$, $q\to0$}}}
      V_q c_{k+q}^\dagger c_{k'-q}^\dagger c_{k'} c_k.
  \]
  So you can see $\mathcal H_\text{off}$ is just $\mathcal H_\text{int}$ with at
  least one momentum that is away from the Fermi surface. We can just keep it in
  the form of $\mathcal H_\text{int}$.

  We choose $\mathcal S$ such that
  \[
    \mathcal H_\text{off} + [\mathcal S, \mathcal H_0] = 0.
  \]
  This means $\mathcal S$ is chosen to cancel the leading-order off-diagonal
  part of $\mathcal H_\text{int}$ under the transformation.

  Then the transformed Hamiltonian becomes
  \[
    \mathcal H' = \mathcal H_0 + \mathcal H_\text{diag}
      + \frac12[\mathcal S, \mathcal H_\text{off}] + \mathcal O(V^3)
  \]
  where the new effective Hamiltonian is block-diagonal up to order $V^2$.

  Constructing $\mathcal S$ explicitly.

  The commutator $[\mathcal S, \mathcal H_0]$ acts as
  \[
    [\mathcal S, \mathcal H_0]
  = \sum_{\alpha\beta} S_{\alpha\beta} (\epsilon_\alpha - \epsilon_\beta)
    c_\alpha^\dagger c_\beta
  \]
  so to satisfy $[\mathcal S, \mathcal H_0] = -\mathcal H_\text{off}$,
  we can write
  \[
    \mathcal S = \sum_{\alpha\beta}
    \frac{(\mathcal H_\text{off})_{\alpha\beta}}
      {\epsilon_\alpha - \epsilon_\beta}.
  \]
  This is the Schrieffer-Wolff generator, which mixes states differing in
  unperturbed energy. For the present problem, $\mathcal S$ connects a bare
  fermion state $c_{k\alpha}^\dagger \ket|\text{FS}>$ to configurations with one
  additional particle-hole pair.
  \paragraph{Now, answer the following questions.}
  \begin{enumext}
    \item Evaluate $[\mathcal S, c_{k\sigma}^\dagger]$;
    \item Transforming the creation operator - ``ressing'' the bare fermion.
    The transformed fermion creation operator is
    \[
      \tilde c_{k\sigma}^\dagger
    = \upe^{\mathcal S} c_{k\sigma}^\dagger \upe^{-\mathcal S}
    = c_{k\sigma}^\dagger + [\mathcal S, c_{k\sigma}^\dagger]
    + \frac12[\mathcal S, [\mathcal S, c_{k\sigma}^\dagger]] + \cdots
    \]
    to the first order in $V$
    \[
      \tilde c_{k\sigma}^\dagger \approx c_{k\sigma}^\dagger
    + [\mathcal S, c_{k\sigma}^\dagger] 
    \]
    \item Compute $\tilde c_{k\sigma}^\dagger \ket|\text{FS}>$.
    \item Compute quasiparticle weight
    $Z_k \equiv |\braket<\text{FS}|\tilde c_k\tilde c_k^\dagger|\text{FS}>|^2$
    from canonical transformation.
  \end{enumext}
\end{problem}